\documentclass[12pt,a4paper]{article}
\usepackage{gvv}
\usepackage{amsmath,amssymb}
\title{\underline{\textbf{2024-1}}}
\date{}
\begin{document}
\maketitle
\begin{enumerate}
    \item Let $f(x)$ be a continuously differentiable function on the interval $(0,\infty)$ such that $f(1) = 2$ and 
\[
    \lim\limits_{t \to x} \frac{t^{10} f(x) - x^{10} f(t)}{t^9 - x^9} = 1
\]
    for each $x > 0$. Then, for all $x > 0$, $f(x)$ is equal to
    \begin{enumerate}
        \item $\frac{31}{11x} - \frac{9}{11} x^{10}$
        \item $\frac{9}{11x} + \frac{13}{11} x^{10}$
        \item $\frac{-9}{11x} + \frac{31}{11} x^{10}$       \item $\frac{13}{11x} + \frac{9}{11} x^{10}$
    \end{enumerate}
 
    \item A student appears for a quiz consisting of only true-false type questions and answers all the questions. The student knows the answers of some questions and guesses the answers for the remaining questions. Whenever the student knows the answer of a question, he gives the correct answer. Assume that the probability of the student giving the correct answer for a question, given that he has guessed it, is $\frac{1}{2}$. Also assume that the probability of the answer for a question being guessed, given that the student’s answer is correct, is $\frac{1}{6}$. Then the probability that the student knows the answer of a randomly chosen question is 
    \begin{enumerate}
        \item $\frac{1}{12}$
        \item $\frac{1}{7}$
        \item $\frac{5}{7}$
        \item $\frac{5}{12}$
    \end{enumerate}
\item Let \( \frac{\pi}{2} < x < \pi \) be such that \( \cot x = \frac{-5}{\sqrt{11}} \). Then 
    \[
    \left( \sin \frac{11x}{2} \right) (\sin 6x - \cos 6x) + \left( \cos \frac{11x}{2} \right) (\sin 6x + \cos 6x)
    \]
    is equal to
    \begin{enumerate}
        \item \( \frac{\sqrt{11} - 1}{2\sqrt{3}} \)
        \item \( \frac{\sqrt{11} + 1}{2\sqrt{3}} \)
        \item \( \frac{\sqrt{11} + 1}{3\sqrt{2}} \)
        \item \( \frac{\sqrt{11} - 1}{3\sqrt{2}} \)
    \end{enumerate}
\item Consider the ellipse \( \frac{x^2}{9} + \frac{y^2}{4} = 1 \). Let \( S(p, q) \) be a point in the first quadrant such that 
    \[
    \frac{p^2}{9} + \frac{q^2}{4} > 1.
    \]
    Two tangents are drawn from \( S \) to the ellipse, of which one meets the ellipse at one end point of the minor axis and the other meets the ellipse at a point \( T \) in the fourth quadrant. Let \( R \) be the vertex of the ellipse with positive \( x \)-coordinate and \( O \) be the center of the ellipse. If the area of the triangle \( \triangle ORT \) is \( \frac{3}{2} \), then which of the following options is correct?
    \begin{enumerate}
        \item \( q = 2, \quad p = 3\sqrt{3} \)
        \item \( q = 2, \quad p = 4\sqrt{3} \)
        \item \( q = 1, \quad p = 5\sqrt{3} \)
        \item \( q = 1, \quad p = 6\sqrt{3} \)
    \end{enumerate}
\item Let $ S= \{ a + b\sqrt{2} : a, b \in \mathbb{Z} \}$, $ T_1= \{ (-1 + \sqrt{2})^n : n \in \mathbb{N} \}$, and $ T_2= \{ (1 + \sqrt{2})^n : n \in \mathbb{N} \}$. Then which of the following statements is (are) TRUE?
    \begin{enumerate}
        \item $ \mathbb{Z} \cup T_1 \cup T_2 \subseteq S $
        \item $ T_1 \cap \left( 0, \frac{1}{2024} \right) = \phi $, where $ \phi $ denotes the empty set.
        \item $ T_2 \cap (2024, \infty) \neq \phi $
        \item For any given $ a, b \in \mathbb{Z} $, $ \cos \left( \pi (a + b\sqrt{2}) \right) + i \sin \left( \pi (a + b\sqrt{2}) \right) \in \mathbb{Z} $ if and only if $ b = 0 $, where $ i = \sqrt{-1} $.
    \end{enumerate}

\item Let $ \mathbb{R}^2 $ denote the Euclidean space. Let 
    $ S = \{ (a, b, c) : a, b, c \in \mathbb{R}, \ ax^2 + 2bxy + cy^2 > 0 \text{ for all } (x, y) \in \mathbb{R}^2 - \{(0,0)\} \} $. Then which of the following statements is (are) TRUE?
    \begin{enumerate}
        \item $ \left( 2, \frac{7}{2}, 6 \right) \in S $
        \item If $ \left( 3, b, \frac{1}{12} \right) \in S $, then $ |2b| < 1 $.
        \item For any given $ (a, b, c) \in S $, the system of linear equations
        \begin{align*}
        ax + by &= 1 \
        bx + cy &= -1
        \end{align*}
        has a unique solution.
        \item For any given $ (a, b, c) \in S $, the system of linear equations
        \begin{align*}
        (a+1)x + by &= 0 \\
        bx + (c+1)y &= 0
        \end{align*}
        has a unique solution.
    \end{enumerate}
\item Let $\mathbb{R}^3$ denote the three-dimensional space. Take two points $P = (1,2,3)$ and $Q = (4,2,7)$. Let $dist(X,Y)$ denote the distance between two points $X$ and $Y$ in $\mathbb{R}^3$. Let  

\[
S = \left\{ X \in \mathbb{R}^3 \; : \; (dist(X,P))^2 - (dist(X,Q))^2 = 50 \right\}
\]

\[
T = \left\{ Y \in \mathbb{R}^3 \; : \; (dist(Y,Q))^2 - (dist(Y,P))^2 = 50 \right\}
\]

Then which of the following statements is (are) TRUE?

\begin{enumerate}
    \item There is a triangle whose area is $1$ and all of whose vertices are from $S$.
    \item There are two distinct points $L$ and $M$ in $T$ such that each point on the line segment $LM$ is also in $T$.
    \item There are infinitely many rectangles of perimeter $48$, two of whose vertices are from $S$ and the other two vertices are from $T$.
    \item There is a square of perimeter $48$, two of whose vertices are from $S$ and the other two vertices are from $T$.
\end{enumerate}
\item Let $a = 3\sqrt{2}$ and $b = \frac{1}{5^{1/6} \sqrt{6}}$. If $x, y \in \mathbb{R}$ are such that  
\[
    3x + 2y = \log_a \left( 18^{5/4} \right)
\]
\[
    2x - y = \log_b \left( \sqrt{1080} \right)
\]
    then $4x + 5y$ is equal to \underline{\quad \quad}.
\item Let $f(x) = x^4 + ax^3 + bx^2 + c$ be a polynomial with real coefficients such that $f(1) = -9$. Suppose that $i\sqrt{3}$ is a root of the equation  
\[
    4x^3 + 3ax^2 + 2bx = 0, \quad \text{where } i = \sqrt{-1}.
\]
    If $\alpha_1, \alpha_2, \alpha_3,$ and $\alpha_4$ are all the roots of the equation $f(x) = 0$, then  
\[
    |\alpha_1|^2 + |\alpha_2|^2 + |\alpha_3|^2 + |\alpha_4|^2
\]
    is equal to \underline{\quad \quad}.
\item Let $S = \left\{ A =  
	\myvec{  
    0 & 1 & c \\  
    1 & a & d \\  
    1 & b & e}
    : a, b, c, d, e \in \{0,1\} \text{ and } |A| \in \{-1,1\}  
    \right\}$,  
    where $|A|$ denotes the determinant of $A$.  
    Then the number of elements in $S$ is \underline{\quad \quad}.
\item A group of 9 students, $s_1, s_2, \dots, s_9$, is to be divided into three teams $X, Y, Z$ of sizes 2, 3, and 4, respectively. Suppose that $s_1$ cannot be selected for the team $X$, and $s_2$ cannot be selected for the team $Y$.  
    Then the number of ways to form such teams is \underline{\quad \quad}.
\item Let  
\[\overrightarrow{OP} = \frac{\alpha -1}{\alpha} \hat{i} + \hat{j} + \hat{k}, \quad
    \overrightarrow{OQ} = \hat{i} + \frac{\beta -1}{\beta} \hat{j} + \hat{k}, \quad
    \overrightarrow{OR} = \hat{i} + \hat{j} + \frac{1}{2} \hat{k}
\]
    be three vectors, where $\alpha, \beta \in \mathbb{R} - \{0\}$ and $O$ denotes the origin.  

    If  
\[
    \left( \overrightarrow{OP} \times \overrightarrow{OQ} \right) \cdot \overrightarrow{OR} = 0
\]
    and the point $(\alpha, \beta, 2)$ lies on the plane  
\[
    3x + 3y - z + l = 0,
\]
    then the value of $l$ is \underline{\quad \quad}.
\item Let $X$ be a random variable, and let $P(X = x)$ denote the probability that $X$ takes the value $x$.  

    Suppose that the points $\left( x, P(X = x) \right)$, where $x = 0,1,2,3,4$, lie on a fixed straight line in the $xy$-plane, and $P(X = x) \neq 0$ for all $x \in \mathbb{R} - \{0,1,2,3,4\}$.  

    If the mean of $X$ is $\frac{5}{2}$, and the variance of $X$ is $\alpha$, then the value of $24\alpha$ is \underline{\quad \quad}.
\item Let $\alpha$ and $\beta$ be the distinct roots of the equation $x^2 + x - 1 = 0$. Consider the set $T = \{1, \alpha, \beta\}$.

For a $3 \times 3$ matrix $M = (a_{ij})_{3 \times 3}$, define $R_i = a_{i1} + a_{i2} + a_{i3}$ and $C_j = a_{1j} + a_{2j} + a_{3j}$ for $i = 1, 2, 3$ and $j = 1, 2, 3$.

Match each entry in List-I to the correct entry in List-II.

\begin{center}
\begin{tabular}{p{12cm}l}
\textbf{List-I} & \textbf{List-II} \\ [8pt]
(P) The number of matrices $M = (a_{ij})_{3 \times 3}$ with all entries in $T$ such that $R_i = C_j = 0$ for all $i, j$, is & [1] 1 \\ [8pt]
(Q) The number of symmetric matrices $M = (a_{ij})_{3 \times 3}$ with all entries in $T$ such that $C_j = 0$ for all $j$, is & [2] 12 \\ [8pt]
(R) Let $M = (a_{ij})_{3 \times 3}$ be a skew-symmetric matrix such that $a_{ij} \in T$ for $i > j$. Then the number of elements in the set 
$\left\{ \begin{pmatrix} x \\ y \\ z \end{pmatrix} : x, y, z \in \mathbb{R}, M = \begin{pmatrix} 0 & a_{12} & 0 \\ -a_{12} & 0 & a_{23} \\ 0 & -a_{23} & 0 \end{pmatrix} \right\}$
is & [3] infinite \\ [8pt]
(S) Let $M = (a_{ij})_{3 \times 3}$ be a matrix with all entries in $T$ such that $R_i = 0$ for all $i$. Then the absolute value of the determinant of $M$ is & [4] 6 \\ [8pt]
 & [5] 0 \\ [8pt]
\end{tabular}
\end{center}

The correct option is
\begin{enumerate}
    \item $P \rightarrow 4$, $Q \rightarrow 2$, $R \rightarrow 5$, $S \rightarrow 1$
    \item $P \rightarrow 2$, $Q \rightarrow 4$, $R \rightarrow 1$, $S \rightarrow 5$
    \item $P \rightarrow 2$, $Q \rightarrow 4$, $R \rightarrow 3$, $S \rightarrow 5$
    \item $P \rightarrow 1$, $Q \rightarrow 5$, $R \rightarrow 3$, $S \rightarrow 4$
\end{enumerate}


\item Let the straight line $y = 2x$ touch a circle with center $(0, a)$, $a > 0$, and radius $r$ at a point $A_1$.
Let $B_1$ be the point on the circle such that the line segment $A_1B_1$ is a diameter of the circle. Let $a + r = 5 + \sqrt{5}$.

Match each entry in List-I to the correct entry in List-II.

\textbf{List-I} \hfill \textbf{List-II}

(P) $a$ equals \hfill (1) $(-2, 4)$

(Q) $r$ equals \hfill (2) $\sqrt{5}$

(R) $A_1$ equals \hfill (3) $(-2, 6)$

(S) $B_1$ equals \hfill (4) 5

\hfill (5) $(2, 4)$

The correct option is
\begin{enumerate}
    \item $ (P) \rightarrow (4) \quad (Q) \rightarrow (2) \quad (R) \rightarrow (1) \quad (S) \rightarrow (3) $
    \item $ (P) \rightarrow (2) \quad (Q) \rightarrow (4) \quad (R) \rightarrow (1) \quad (S) \rightarrow (3) $
    \item $ (P) \rightarrow (4) \quad (Q) \rightarrow (2) \quad (R) \rightarrow (5) \quad (S) \rightarrow (3) $
    \item $ (P) \rightarrow (2) \quad (Q) \rightarrow (4) \quad (R) \rightarrow (3) \quad (S) \rightarrow (5) $
\end{enumerate}

\item Let $\gamma\in\mathbb{R}$ be such that the lines
\[L_{1}:\frac{x+11}{1}=\frac{y+21}{2}=\frac{z+29}{3}\]
and
\[L_{2}:\frac{x+16}{3}=\frac{y+11}{2}=\frac{z+4}{\gamma}\]
intersect. Let $R_{1}$ be the point of intersection of $L_{1}$ and $L_{2}$. Let $O=(0,0,0)$ and $\hat{n}$ denote a unit normal vector to the plane containing both the lines $L_{1}$ and $L_{2}$.

Match each entry in List-I to the correct entry in List-II.

\begin{tabular}{ll}
List-I & List-II \\
(P) $\gamma$ equals & (1) $-\hat{i}-\hat{j}+\hat{k}$ \\
(Q) A possible choice for $\hat{n}$ is & (2) $\sqrt{\frac{3}{2}}$ \\
(R) $\vec{OR}_{1}$ equals & (3) 1 \\
(S) A possible value of $\vec{OR}_{1}\cdot\hat{n}$ is & (4) $\frac{1}{\sqrt{6}}\hat{i}-\frac{2}{\sqrt{6}}\hat{j}+\frac{1}{\sqrt{6}}\hat{k}$ \\
& (5) $\sqrt{\frac{2}{3}}$
\end{tabular}

The correct option is
\begin{enumerate}
    \item $ (P) \rightarrow (3) \quad (Q) \rightarrow (4) \quad (R) \rightarrow (1) \quad (S) \rightarrow (2) $
    \item $ (P) \rightarrow (5) \quad (Q) \rightarrow (4) \quad (R) \rightarrow (1) \quad (S) \rightarrow (2) $
    \item $ (P) \rightarrow (3) \quad (Q) \rightarrow (4) \quad (R) \rightarrow (1) \quad (S) \rightarrow (5) $
    \item $ (P) \rightarrow (3) \quad (Q) \rightarrow (1) \quad (R) \rightarrow (4) \quad (S) \rightarrow (5) $
\end{enumerate}
\item Let $f: \mathbb{R} \to \mathbb{R}$ and $g: \mathbb{R} \to \mathbb{R}$ be functions defined by
\[
f(x) =
\begin{cases} 
x^2 \sin \frac{1}{x}, & x \ne 0, \\
0, & x = 0,
\end{cases}
\]
and 
\[
g(x) = 
\begin{cases} 
1-2x, & 0 \le x \le \frac{1}{2}, \\
0, & \text{otherwise}.
\end{cases}
\]
Let $a, b, c, d \in \mathbb{R}$. Define the function $h: \mathbb{R} \to \mathbb{R}$ by
\[
h(x) = af(x) + b(g(x) + g(1-x)) + c(x - g(x) + g(1-x)) + d g(x), \quad x \in \mathbb{R}.
\]

Match each entry in List-I to the correct entry in List-II.

\begin{tabular}{ll}
\multicolumn{1}{c}{List-I} & \multicolumn{1}{c}{List-II} \\
(P) If $a=0, b=1, c=0,$ and $d=0$, then & (1) $h$ is one-one. \\
(Q) If $a=1, b=0, c=0,$ and $d=0$, then & (2) $h$ is onto. \\
(R) If $a=0, b=0, c=1,$ and $d=0$, then & (3) $h$ is differentiable on $\mathbb{R}$. \\
(S) If $a=0, b=0, c=0,$ and $d=1$, then & (4) The range of $h$ is $[0,1]$. \\
 & (5) The range of $h$ is $\{0,1\}$. 
\end{tabular}
The correct option is
\begin{enumerate}
    \item $ (P) \to (4) \quad (Q) \to (3) \quad (R) \to (1) \quad (S) \to (2) $
    \item $ (P) \to (5) \quad (Q) \to (2) \quad (R) \to (4) \quad (S) \to (3) $
    \item $ (P) \to (5) \quad (Q) \to (3) \quad (R) \to (2) \quad (S) \to (4) $
    \item $ (P) \to (4) \quad (Q) \to (2) \quad (R) \to (1) \quad (S) \to (3) $
\end{enumerate} 
\end{enumerate}
\end{document}
